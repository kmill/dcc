\documentclass[11pt]{article} 

\usepackage{amsmath, amsfonts, amssymb, amsthm}
\usepackage{fullpage}

\title{6.035 Project 2}
\author{Kyle Miller, Alec Thomson, Patrick Hulin, Steven Valdez}
\begin{document} 
\maketitle

\section {Overview}

The Purpose of this project was to extend the semantic checker
produced at the end of the last project to actually generate correct
x86 assembly code. 

Our Code Generator follows a four step process. The first step is to
translate the Hybrid AST and Symbol Table produced as the output of
the previous project into a Mid-Level IR that represents the code as a
graph of Basic Blocks. The second step is to translate the mid IR into
a Low-Level IR that translates the higher level instructions used in
the mid IR to low level assembly instructions. The low IR initially
makes use of a number of ``Symbolic Registers'', so the next step of
code generation is to use a basic Register Allocator to replace these
symbolic registers with real registers and memory locations. Finally,
the low IR with all symbolic registers allocated is translated
directly into assembly language. 

The rest of this document outlines each of these steps in detail. The
mid level IR is described in Section~\ref{sec:midir}. The low level IR
is described in Section~\ref{sec:lowir}. The register allocator is
described in Section~\ref{sec:regalloc}. The assembly generator is
described in Section~\ref{sec:assembly}. 

\section{Division of Work} 
\label{sec:division}

The work was divided as follows. Kyle Miller designed and wrote the
mid IR and the low IR and wrote all of the translation functions
between them. Kyle also implemented several optimizations. Alec
Thomson wrote an initial assembly template that provided some skeleton
code for eventual assembly generation. Alec also wrote the simple
register allocator. Patrick Hulin and Steven Valdez wrote the code to translate the
low ir directly into assembly code. Each member of the team spent a
considerable amount of time debugging the final product. 

\section{Assumptions about Spec} 
\label{sec:assumptions}

The following were assumptions we made about the Decaf spec while we
wrote the code generator: 

\begin{itemize}
\item The Array index expression is evaluated before the assignment expression. 
\item Binary operators are evaluated left to right. 
\item Method arguments are evaluated right to left. 
\end{itemize}

\section {Mid IR}
\label{sec:midir}

TODO: Totally put an image of one of the graphs created by graphviz here 

\section {Low IR} 
\label{sec:lowir}

TODO: Totally put an image of one of the graphs created by graphviz here 

\section {Register Allocation} 
\label{sec:regalloc}

all symbolic registers from the Low IR. It works by maintaining state
through the use of the State Monad to keep track of where previously
encountered symbolic registers are located on the stack. When it
encounters an unallocated symbolic register, it gives that register
the next location on the stack and adds that mapping to its state. It
subsequently updates the next location on the stack as well. 

The register allocator also generates additional code to be inserted
into the low ir to allow for the loading and storing of memory
locations that correspond to the symbolic registers. At the moment,
this code is very basic and very inefficient. Individual values are
often shuffled through many more registers and memory locations than is
necessary. Despite this inefficiency, the register allocator is useful
because of its simplicity. The framework also exists now to produce a
more complex and more efficient register allocator in the future. 

Problems that arose while implementing the register allocator
primarily consisted of incorrectly mapping symbolic registers to
temporary registers during loads and stores. Additional code needed to
be inserted to account for indirect loads and stores of symbolic
registers. All of this code is documented and included in RegisterAllocator.hs

\section {Assembly Generation}  
\label{sec:assembly}
The assembly generation was mainly accomplished via a bunch of methods which
translated from the Low IR to lists of strings of instructions. These methods
were then combined to build a bigger part of the overall representation. While
we probably caould have designed a better mechanism to make the type system
check more of our code's safety, most of the errors we made at this stage were
far beyond the type system's abilities. With our design decisions, the code
itself was very straightforward to write, but the debugging was a total pain,
probably taking 70\% of the time. It's not clear that we could have avoided
this. Most of the problems were due to an insufficient understanding or
incorrect implementation of the calling convention.




\end{document}
