\documentclass[11pt]{article} 

\usepackage{amsmath, amsfonts, amssymb, amsthm}
\usepackage{fullpage}

\title{6.035 Project 2}
\author{Kyle Miller, Alec Thomson, Patrick Hulin, Steven Valdez}
\begin{document} 
\maketitle

\section {Overview}

The Purpose of this project was to extend the semantic checker
produced at the end of the last project to actually generate correct
x86 assembly code. 

Our Code Generator follows a four step process. The first step is to
translate the Hybrid AST and Symbol Table produced as the output of
the previous project into a Mid-Level IR that represents the code as a
graph of Basic Blocks. The second step is to translate the mid IR into
a Low-Level IR that translates the higher level instructions used in
the mid IR to low level assembly instructions. The low IR initially
makes use of a number of ``Symbolic Registers'', so the next step of
code generation is to use a basic Register Allocator to replace these
symbolic registers with real registers and memory locations. Finally,
the low IR with all symbolic registers allocated is translated
directly into assembly language. 

The rest of this document outlines each of these steps in detail. The
mid level IR is described in Section~\ref{sec:midir}. The low level IR
is described in Section~\ref{sec:lowir}. The register allocator is
described in Section~\ref{sec:regalloc}. The assembly generator is
described in Section~\ref{sec:assembly}. 

\section{Division of Work} 
\label{sec:division}

The work was divided as follows. Kyle Miller designed and wrote the
mid IR and the low IR and wrote all of the translation functions
between them. Kyle also implemented several optimizations. Alec
Thomson wrote an initial assembly template that provided some skeleton
code for eventual assembly generation. Alec also wrote the simple
register allocator. Patrick Hulin and Steven Valdez wrote the code to translate the
low ir directly into assembly code. Each member of the team spent a
considerable amount of time debugging the final product. 

\section{Assumptions about Spec} 
\label{sec:assumptions}

The following were assumptions we made about the Decaf spec while we
wrote the code generator: 

\begin{itemize}
\item The Array index expression is evaluated before the assignment expression. 
\item Binary operators are evaluated left to right.
\item Method arguments are evaluated right to left.
\end{itemize}

The first two assumptions are fairly arbitrary, but the third was
chosen in this way because it is easier generate assembly instructions
in right-to-left order because stack-based arguments are stored in
pop-friendly order.

\section {Mid IR}
\label{sec:midir}

We decided on having two levels of intermediate representations
between the AST and the assembly code.  The first is the Mid IR, which
is the subject of this section, and the second is the Low IR, which is
the subject of the next.  Each of these IRs are graph structures where
each vertex is a sequence of instructions culminating in some kind of
jump, and each edge represents the connection from the jumper to the
jumpee.  The distinction between the two is that the Mid IR deals with
symbolic variables where the Low IR deals with registers only, and
also the Mid IR has no idea about the \texttt{d(r,r,s)}-style
addressing schemes, where the Low IR does.  We took the
instruction-level design of our IR from Muchnick.

The conversion from the AST to the Mid IR was accomplished by a
continuation-passing-style algorithm where each continuation was the
ID of a basic block to jump to in the circumstances of 1) success, 2)
breaking from a loop, and 3) continuing the next iteration of a loop.
Inside expressions, the continuation was limited to only success, but
an additional continuation, the temporary variable in which to store
the result of a subexpression, was also passed around.

The continuation style made it straightforward to implement all of the
control structures.  The \texttt{if} statement gives a taste of the general method:
\begin{verbatim}
statementToMidIR env s c b (HIfSt _ pos expr cons malt)
    = do cons' <- statementToMidIR env s c b cons
         alt' <- case malt of
                   Just alt -> statementToMidIR env s c b alt
                   Nothing -> return s
         tvar <- genTmpVar
         block <- newBlock [] (IRTest (OperVar tvar)) pos
         addEdge block True cons'
         addEdge block False alt'
         e <- expressionToMidIR env block tvar expr
         return e
\end{verbatim}
where each statement conversion has the type
\begin{verbatim}
statementToMidIR :: IREnv
                 -> Int -- ^ BasicBlock on success
                 -> Int -- ^ BasicBlock on continue
                 -> Int -- ^ BasicBlock on break
                 -> HStatement a -> State MidIRState Int
\end{verbatim}
Remarkably, \texttt{break} and \texttt{continue} have rather simple
definitions:
\begin{verbatim}
statementToMidIR env s c b (HBreakSt _ pos)
    = return b
statementToMidIR env s c b (HContinueSt _ pos)
    = return c
\end{verbatim}

To remove the need to maintain lexical environments, we did alpha
renaming of the variables in functions so that variables have the same
name if and only if they represent the same variable (i.e., are in the
same lexical scope).

No effort was made to generate reasonable Mid IR code output since we
are assuming we'll be able to remove the junk in later optimization
passes.

We did make a minor effort toward optimizing the IR graph.  We made a
small graph rewrite algoritm framework to be able to convert an IR
graph into normal form where each basic block is as big as possible.
This algoirthm gave unreachable block removal for free.

\section {Low IR} 
\label{sec:lowir}

The Low IR is similar to the Mid IR except it has a better idea of the
x86 architecture including its addressing schemes.  We let the Low IR
use symbolic registers to simulate a machine with infinitely many
registrs so that we don't need to deal with the hard truth that there
are only 16 registers on our CPU.

The translation from the Mid IR involved keeping track of a table of
variable name to symbolic register.  Each basic block is converted one
at a time, statement by statement inside of them.

We tried doing some basic optimizations at this stage involving
converting the Low IR to a tree-based low ir (where statements in the
basic blocks are tree structures, and converting back using tree
rewrite rules.  This produces nice-looking low ir code which has
optimized out most of the spurious register created by the previous
steps.  However, we have disabled this feature for now since it
involved too much untested code, and it produced code which failed on
\texttt{12-huge.pdf} (however, last we checked, each of the other
tests worked fine).  Tho optimizer can be enabled by passing in the
\texttt{--opt} argument.

The intermediate representations each emit \texttt{graphviz} code
which can be translated into a graphical representation by \texttt{dot
  -Tpdf -o\textit{outfile} \textit{infile}}.  We would include such
images in this document to show them off, but but they are generally
much too large to make any sense of from a distance.

\section {Register Allocation} 
\label{sec:regalloc}

All of the symbolic registers from the Low IR. The allocator works by
maintaining state through the use of the State Monad to keep track of
where previously encountered symbolic registers are located on the
stack. When it encounters an unallocated symbolic register, it gives
that register the next location on the stack and adds that mapping to
its state. It subsequently updates the next location on the stack as
well.

The register allocator also generates additional code to be inserted
into the low ir to allow for the loading and storing of memory
locations that correspond to the symbolic registers. At the moment,
this code is very basic and very inefficient. Individual values are
often shuffled through many more registers and memory locations than is
necessary. Despite this inefficiency, the register allocator is useful
because of its simplicity. The framework also exists now to produce a
more complex and more efficient register allocator in the future. 

Problems that arose while implementing the register allocator
primarily consisted of incorrectly mapping symbolic registers to
temporary registers during loads and stores. Additional code needed to
be inserted to account for indirect loads and stores of symbolic
registers. All of this code is documented and included in
\texttt{RegisterAllocator.hs}

\section {Assembly Generation}  
\label{sec:assembly}
The assembly generation was mainly accomplished via a bunch of methods
which translated the Low IR to lists of strings of instructions. These
methods were then combined to build a bigger part of the overall
representation. While we probably caould have designed a better
mechanism to make the type system check more of our code's safety,
most of the errors we made at this stage were far beyond the type
system's abilities. With our design decisions, the code itself was
very straightforward to write, but the debugging was a total pain,
probably taking 70\% of the time. It's not clear that we could have
avoided this. Most of the problems were due to an insufficient
understanding of or incorrect implementation of the calling
convention.




\end{document}
