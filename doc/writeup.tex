\documentclass[11pt]{article}

\usepackage{amsmath,amsfonts,amssymb,amsthm} %,latexsym,mathrsfs,bbm}
\usepackage{fullpage}

\title{6.035 Project 0}
\author{Kyle Miller}
\date{21 Feb 2012}
\begin{document}
\maketitle

This is an overview of Project 0, which was a project about writing a
small program.  In particular, the goal was to make a scanner and
parser for the Decaf language.

The decision was made very early on (i.e., last year) to use Haskell
as the language to build the compiler.  This was due to varying
reasons.  An attempt to solidify these reasons would have as some of
them 1) the author has been meaning to actually learn how to program
in Haskell, 2) the language seems to be very well-suited to designing
domain-specific languages, which seems to the currently naive author
useful for building a compiler, and 3) monads\footnote{etc.}.

The specification was followed very closely; in fact, it was followed
to such a maddening degree that ANTLR quirks were coded into the
scanner and the output was made character perfect.  There were two
additions made to the project specification.  The first was to make a
command line flag called \texttt{--compat} which, when given, ensures
that the output of the program matches the required output perfectly.
The second was to implement an abstract syntax tree and output it when
\texttt{--compat} mode is not given.  The rationale for having an
abstract syntax tree is for being able to test, for instance, whether
order of operations actually works.

There was one modification made to the specification, and that was to
make it so command line arguments begin with \texttt{--} if they are
in long form and \texttt{-} if they are in short form, like any other
program, rather than having all arguments begin with \texttt{-}.  The
rationale for this will be discussed in section
\ref{sec:comm-line-interf}.

\section{Invocation}
\label{sec:invocation}

To build the project, run
\begin{verbatim}
$ cabal configure
$ cabal build
\end{verbatim}
There is a symbolic link \texttt{dcc} in the root of the archive which
runs the built program.  The test case script in
\texttt{tests/scanner/test.sh} and \texttt{tests/parser/test.sh} have
been modified to be able to find \texttt{dcc}.

\section{Design overview}
\label{sec:design-overview}

This section briefly describes the various parts of the scanner and
parser and their design.

\subsection{Command line interface}
\label{sec:comm-line-interf}

The downside to using Haskell as far as amount of programming work
goes is that there was no CLI or supporting framework, so the it had
to be built.  For a CLI, it seemed like many real Haskell applications
used \texttt{System.Console.GetOpt}, so the author decided to use it
as well.  The interface is not complicated---you give it a table of
the possible options as well as functions to convert the parsed
options into some internal data structure.  And, not only does it do
the parsing, but it also is able to automatically generate a usage
page. Also, looking at the code, it's almost unfair comparing
\texttt{CLI.java} with \texttt{CLI.hs} since \texttt{GetOpt} is such a
nice DSL.

The ``problem'' with \texttt{GetOpt}, as far as the 6.035
specification is concerned, is that it interprets a single
``\texttt{-}'' as meaning the start of a short form of an argument,
reserving long forms for ``\texttt{--}'' prefixes.  It seems
\texttt{GetOpt} was designed this way since then multiple single-hypen
arguments can be concateneted, dropping off hyphens (so, ``\texttt{-a
  -u -x}'' can be written as ``\texttt{-aux}''). It would be a very
ugly hack to try to convince \texttt{GetOpt} to like single-hyphen
long-form arguments, so the decision was made not to do this.  To be
more specific, three ways which come to mind which almost work but are
unacceptable are
\begin{enumerate}
\item textually replace all single-hyphen arguments with their
  doubly-hypenated counterparts, duplicating the efforts of
  \texttt{GetOpt} to parse arguments and possibly losing the error
  handling;
\item textually replace instances of single hyphen with two hyphens
  and adding single-character ``long-form arguments'', losing
  \texttt{GetOpt}'s nice automatic usage page; or
\item textually replace all doubly hyphenated arguments with a single
  hyphen and its first character, losing \texttt{GetOpt}'s error
  handling mechanism.
\end{enumerate}
If forced, the author might begrudgingly implement the first one and
make a compiler flag that would compiled in this feature.

\subsection{Scanner}
\label{sec:scanner}

There were two possible tools the author knew about for building a
scanner: Alex and Parsec.  Alex is a lexer generator which takes a
specification, like ANTLR, and produces Haskell code which turns
strings into token streams.  Parsec is a parser combinator library
which lets one build parsers for any kind of stream (that is, a list
of some kind of data, such as characters) by composing them from other
parsers.  Since Parsec implements regular expression operations, it
would be able to produce at least what Alex could.  For this reason
along with the author's inherent distrust for code generators, Parsec
was chosen.

The design of the \texttt{Token} data constructor was chosen to
simplify the code of the scanner.  In particular, the constructor was
defined with arguments in the order of \texttt{TokenType},
\texttt{SourcePos}, \texttt{String} (where \texttt{TokenType} is
something like \texttt{CharLiteral} or \texttt{Identifier}), to
facilitate using applicative functor notation.  For instance, the
general function for making token objects using the current scanner
state was able to be written like
\begin{verbatim}
makeToken :: TokenType -> Scanner String -> Scanner Token
makeToken restype parser = Token restype <$> getPosition <*> parser
\end{verbatim} %$

One hangup with using Parsec instead of the 6.035-prescribed ANTLR was
that Parsec, upon encountering a scanning error (which is consuming
input and entering a state in which the scanner cannot proceed), will
immediately halt and deliver a useful error message, rather than
skipping the character, collecting the error, and continuing scanning,
as ANTLR scanners would, despite the fact that scanner errors tend to
occur in strings, and continuing after a messed-up string will just
scan everything in the string as out of it, and vice versa.

Since the specification required collecting all errors, this
difficulty with Parsec needed to be solved.  First, we need a quick
overview of how Parsec works.  It is a top-down parser whose
individual parsing functions compute a value of a type like
\texttt{Consumed (Reply a)} where
\begin{verbatim}
data Consumed a = Consumed a | Empty !a
data Reply a = OK a ParserState | Error ParseError
\end{verbatim}
where the \texttt{Consumed} type represents whether the parser
consumed input in the process of parsing (which occurs due to
lookahead) and the \texttt{Reply} type represents whether the parser
actually was able to parse.  To concatenate parsers, the rule is that
if the first parser comes to an error then the entire parser results
in that error, and whether the concatenated parser consumed anything
is a function of whether either consumed anything.  Interestingly, to
make a parser which parses either of the parsers, which we'll call the
choice parser, the first is executed, and if it fails without having
consumed input, then the second parser is executed.  The choice parser
fails if either the second parser fails or if the first parser fails
after having consumed input.  This behavior for the choice parser
makes Parsec efficient by removing implicit lookahead.

With this in mind, the problem is still to make it so the scanner can
continue after failure with Parsec.  There are a few options.
\begin{itemize}
\item Make a function which takes a parser and catches \texttt{Error}.
  The problem with this approach is that Parsec forgets its state in
  the event of an error to prevent memory leaks!  While it is true
  that the \texttt{ParseError} contains the position in the file where
  the error occurred, taking the line and column data and restarting a
  parser from there would be annoying.  Furthermore, to get a
  character-perfect replication of the test cases, we would need to
  put in code which would disrupt normal Parsec error messages, which
  would be annoying in the event we would want to turn them off.
\item Make a parser which could inspect the current Parsec errors
  (which are stored in \texttt{ParserState}) and create a special
  error token object which would represent the error.  This would be
  put at the end of a choice parser so that Parsec would think the
  parse was a success since an object was returned and then not stop
  parsing.  There are two problems: a) Parsec's library is watertight
  and such information cannot be had by user-written parsers, and b)
  the logic to short-circuit when getting a error token object would
  be annoying.
\item Do something like the previous option, but wrap the parser in a
  monad transformer which could do such short circuiting
  automatically.
\end{itemize}
We went with the third option because it seemed the cleanest.  The
transformer is in \texttt{Control.Exceptional}, which was written
especially for the scanner.  It is essentially \texttt{EitherT} except
with different names because older versions of the Prelude didn't have
instances for \texttt{Either e} being a monad.  Scanners which could
have a scanner error are put in the \texttt{EitherT}, where the
\texttt{Left} holds a token representing the error.

The decision was made to make the above mechanism disabled by default
and to be able to turn on collecting errors using the
``\texttt{--compat}'' argument due to the taste of the author.  This
is accomplished by having some state in the parsers which makes the
error-building parser evaluate to \texttt{mzero} if the flag is not
set.

There is an oddity in ANTLR where unexpected tabs increment the column
number by 8 and unexpected newlines increment the column number by 2.
It doesn't increment the line number!  This oddity was accounted for
in ``\texttt{--compat}'' mode.

\subsection{Parser}
\label{sec:parser}

Having had success with Parsec, the parser was also written with the
library.  The parser construction is straightforward and looks
somewhat like the grammar itself. Applicative functor notation was a
great benefit to parser construction.  For instance, to define a
``for'' statement one may write
\begin{verbatim}
forSt :: DParser Statement
forSt = ForSt <$> getPosition
        <* dkeyword "for" <* dopenp
        <*> ident <* dkeyword "="
        <*> expr <* dkeyword ";"
        <*> expr <* dclosep
        <*> block
\end{verbatim} %$

Every once in a while, up to three symbols of lookahead were used to
figure out which branch of the parser to take.  Wherever lookahead
occurs in the source, there is a comment acknowledging this fact.

An interesting part of the parser is the definition of expressions.
Mirroring the specification, the operators were defined by
\begin{verbatim}
ops :: [DParser Expr -> DParser Expr]
ops = [ unOps ["-"]
      , unOps ["!"]
      , binOps ["*", "/", "%"]
      , binOps ["+", "-"]
      , binOps ["<", "<=", ">=", ">"]
      , binOps ["==", "!="]
      , binOps ["&&"]
      , binOps ["||"]
      ]
\end{verbatim}
and then code was written to tell the parser on each line about the
parser on the previous line, where the first parser was told about the
parsers for method calls, literals, and further expressions nested in
parentheses.  This makes it easy to add more operators to the language
at a later date at any precedence level while also making sure tedious
code did not have to be written for defining expressions.

A benefit of using Parsec is good at giving lists of expected tokens.
For instance,
\begin{verbatim}
"./illegal/illegal-16" (line 3, column 9):
    if () { // no condition
        ^
unexpected keyword ")"
expecting unary operator, identifier, "callout", literal or "("
\end{verbatim}
(the part which points to a line of source with a carat was written
for this project).  There is currently one semi-bug with the parser,
and that is a problem in reporting errors when lookahead is involved.
The following is an example of this:
\begin{verbatim}
"./illegal/illegal-12" (line 7, column 11):
    int a b;	// missing comma
          ^
unexpected identifier "b"
expecting "," or ";"
\end{verbatim}
The ``expecting'' line should have also had an open parenthesis, but
because two-symbol lookahead was used to determine whether the parser
was trying to find a method or field declaration, the parser forgot
that an open parenthesis was used in the decision.  At the moment,
it's unclear to the author how to fix this error message.

\section{Discovered patterns}
\label{sec:discovered-patterns}

This section documents various code patterns which were either
discovered by trial and error or from other sources in the event that
they might be useful to use in the future.

The first is \texttt{foldl (flip id) s l} where \texttt{s} is of type
$a$ and \texttt{l} is of type $[a\to a]$.  This iteratively applies
the functions in \texttt{l} with \texttt{s} as the initial input.  So,
what is the function \texttt{flip id}?  We'll just do a partial
evaluation to figure it out, where the symbol $\mapsto$ refers to the
lambda abstraction and is right-associative:
\begin{align*}
  \text{\texttt{flip\ id}}
  &=(f\mapsto x\mapsto y\mapsto f y x)(x\mapsto x)\\
  &=(x\mapsto y\mapsto (x\mapsto x) y x)\\
  &=(x\mapsto y\mapsto y x)\\
  &=\text{\texttt{flip\ (\$)}}.
\end{align*}
That is, \texttt{flip id} is the reverse apply operator.  So,
\texttt{flip id x} is a function which takes a function and applies
the argument \texttt{x} to it.  This occurs in the wild both in the
CLI interface for evaluating the parsed arguments and in the parser
for sequencing the parsers.

The second is for a ``double bind,'' which for binding two monads to a
function at the same time.  In particular, we are talking about a
function of the type \texttt{Monad m => (a -> b -> m c) -> m a -> m b
  -> m c}.  A straightforward way to define such a function is
\begin{verbatim}
dbind f ma mb = do a <- ma
                   b <- mb
                   f a b
\end{verbatim}
but the author doesn't think that is very elegant.  Recalling the fact
that the reverse bind operator, of type \texttt{Monad m => (a -> m b)
  -> m a -> m b}, can be written as \texttt{join (fmap f a)} for
\texttt{f =<< ma} since \texttt{fmap f a} is of type \texttt{Monad m % >>
  => m (m b)} and \texttt{join} removes one layer of monads, we can
try some sort of analogue using fmap and join.  It ends up being
\begin{verbatim}
dbind f ma mb = join (f <$> ma <*> mb)
\end{verbatim} % $
where \texttt{(<\$>)} is infix \texttt{fmap} and \texttt{(<*>)} is the
composition operator for applicative functors, defined for monads to
be
\begin{verbatim}
mf <*> ma = do f <- mf
               a <- ma
               return $ f a
\end{verbatim} % $
or, mildly more sneakily,
\begin{verbatim}
mf <*> ma = mf >>= (\f -> a >>= (return . f))
\end{verbatim} % $

\end{document}
